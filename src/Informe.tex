\documentclass[12pt,a4paper]{article}
\usepackage{graphics}
\includegraphics[scale=15][imagen1.eps]
\author{Miguel Hernández}
\title{Aproximación de $\pi$}
\begin{document}
\maketitle
\talbeofcontents

\begin{abstract}
$\pi$ (pi) es la relación entre la longitud de una circunferencia y su diámetro, en geometría euclidiana. Es un número irracional y una de las constantes matemáticas más importantes. Se emplea frecuentemente en matemáticas, física e ingeniería.
$\pi$
\end{abstract}
Este artículo se dedica a la explicación y a la divulgación del dicho número. La información no es propia, solo es para uso de la práctica.
\section{Cálculo y definición de $\pi$}
Frente a las muchas formas de computar el número $\pi$, en esta asignatura hemos utilizado una implementación que nos permite calcular y aproximar el número mediante un algoritmo informático.
\begin{sangria}
\subsection{Definiciones previas}

Euclides fue el primero en demostrar que la relación entre una circunferencia y su diámetro es una cantidad constante.15 No obstante, existen diversas definiciones del número \pi, pero las más común es:

    $\p$i es la relación entre la longitud de una circunferencia y su diámetro.

Por tanto, también \pi es:

    El área de un círculo unitario (de radio unidad del plano euclídeo).
    El menor número real x positivo tal que $\sin$(x) = 0.

También es posible definir analíticamente \pi; dos definiciones son posibles:

    La ecuación sobre los números complejos e^{ix}+1=0 admite una infinidad de soluciones reales positivas, la más pequeña de las cuales es precisamente $\pi$ (véase identidad de Euler).
    La ecuación diferencial S''(x)+S(x)=0 con las condiciones de contorno S(0)=0, S'(0)=1 para la que existe solución única, garantizada por el teorema de Picard-Lindelöf, es un función analítica (la función trigonométrica $\sin$(x)) cuya raíz positiva más pequeña es precisamente $\pi$.

Así este número posee estructura de número irracional, sus 50 primeras cifras son:

    $\pi$ = 3.1415926535897932384626434323279508841971693937510
    
Para el cálculo de dicho número, como en las prácticas usaremos:
   $\sum_{i=n} f(xi)$ con $\f(x)=4/(i+x^2)$)$ y xi=(i-1/n)/n$

\subsection{Algoritmo}
\end{sangria}
\begin{tabular}{|l|}%incluimos una tabla con nun numero de tablas que vienen introducidos 
\hline
import sys\newline
PI=3.14159265358979323643380288\newline
def funcion(n):\newline
  if (n!=0):\newline
    suma=0.0\newline
    for i in range(1,n+1):\newline
      a=(i-0.5)/n\newline
      b=float(i)/n\newline
      xi=(i-0.5)/n	\newline
      fxi=4.0/(1.0+xi*xi)\newline
      suma=suma+fxi\newline
    resultado=suma/n\newline
    return resultado\newline
    
if __name__=="__main__":\newline

 if(len(sys.argv)==1) or (len(sys.argv)==2):\newe=15][imagen1.eps]line
  print '\nNo se han introducido datos,usaremos los valores por defecto de la funcion:\newline
  Resultados:'\newline
  n=10\newline
  m=10\newline
 else:\newline
  n=int(sys.argv[1])\newline
  m=int(sys.argv[2])\newline
\newline
 l=[]\newline
 for j in range(1, m+1):\newline
  aproximacion=funcion(j*n)\newline
  l=l+[aproximacion]\newline
 print l \hline\newline
 PARTE QUE NO CORRESPONDE A UNA FUNCIÓN
 \hline
 import fpython\newline

t_upla=(10,100,1000,10000,100000.1000000,10000000,100000000,1000000000,10000000000,100000000000,100000000000)\newline#Crash en 4; para 5 o + valores; no
for i in t_upla:\newline
  ap=fpython.funcion(i)\newline
  print ap\hline
\end{tabular}

\section{Conclusiones}
...y aqui termina.
\bibitem[Wikipedia]{Número $\pi$}
\cite{Numero $\pi$}

\footnote{Resumen de la práctica}

\end{document}